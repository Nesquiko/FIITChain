\section{Phase 1 - Simple Coin}\label{phase1}

\subsection{Assignment}

Centralized authority FIIT accepts transactions from users. Implement logic
for processing transactions into a ledger. FIIT groups transactions into 
into a pseudo blockchain. In each block, FIIT will receive list of transactions
which are validated, applied and subset of valid ones is returned.

These transactions can depend on each other, there may be double-spends, and
otherwise invalid transactions.

In this phase, implement a UTXO FIIT chain, which in each epoch takes a list
of proposed transactions, validates them, applies valid ones to its internal
state and returns a subset of valid ones. The size of the subset isn't defined.

Also implement a version in which, not only validity is checked, but also transactions
are processed in order to maximize received fees.

\subsection{Implementation}

Implementation for this phase is in submodule \emph{fiitcoin}. This submodule
is a Rust library crate, so I could easily reuse it in next phases with importing
it as a dependency.

Required handlers are in rust file \emph{fiitcoin/src/handler.rs}. Both implement
common trait \texttt{TxHandler}:

\pagebreak
\begin{lstlisting}[language=Rust, style=boxed, caption={TxHandler}]
pub trait TxHandler<'a> {
    /// Each epoch accepts unordered vector of proposed transactions.
    /// Checks validity of each, internally updates the UTXO pool, and
    /// returns vector of valid ones.
    ///
    /// # Beware
    /// Transactions can be dependent on other ones. Also, multiple
    /// transactions can reference same output.
    fn handle(&mut self, possible_txs: Vec<&'a Tx>) -> Vec<&'a Tx>;

    /// Returns reference to internal pool
    fn pool(&self) -> &UTXOPool;

    /// Returns mutable reference to internal pool
    fn pool_mut(&mut self) -> &mut UTXOPool;

    /// Moves internal pool, while consuming self
    fn move_pool(self) -> UTXOPool;

    /// Checks if:
    ///     1. All UTXO inputs are in pool
    ///     2. Signatures on inputs are valid
    ///     3. No UTXO is used more than once
    ///     4. Sum of outputs is not negative
    ///     5. Sum of inputs >= Sum of outputs
    fn is_tx_valid(&self, tx: &Tx) -> bool;

    /// Filters independent txs from dependent ones,
    /// applies them and returns both sets
    fn handle_independent(
        &mut self,
        txs: Vec<&'a Tx>
    ) -> (Vec<&'a Tx>, Vec<&'a Tx>);

    /// Applies given tx to the internal pool
    fn apply_tx(&mut self, tx: &Tx);


    fn is_input_in_pool(&self, input: &Input) -> bool;
}
\end{lstlisting}

Methods \texttt{is\_tx\_valid}, \texttt{handle\_independent}
\texttt{apply\_tx}, \texttt{is\_input\_in\_pool} have
default implementations in the trait declaration, because all of them are used
in both handlers. Methods \texttt{handle}, \texttt{pool}, \texttt{pool\_mut},
and \texttt{move\_pool} are implemented in each handler.

The \texttt{is\_tx\_valid} function is self explanatory, and all checks, which
are performed are listed in its doc comment.

The \texttt{handle\_independent} function is used to separate transactions,
which are not dependent on transactions in the currently proposed list and can be
applied right away, from invalid and dependent transactions. The valid independent
ones are also applied to the internal state.

\begin{lstlisting}[language=Rust, style=boxed, caption={handle\_independent}]
/// Filters independent txs from dependent ones,
/// applies them and returns both sets
fn handle_independent(
    &mut self,
    txs: Vec<&'a Tx>
) -> (Vec<&'a Tx>, Vec<&'a Tx>) {
    let mut handled = vec![];
    let mut dependent = vec![];
    let tx_set: HashSet<[u8; 32]> = txs.iter().map(|&tx| tx.hash()).collect();

    for &tx in txs.iter() {
        if tx.inputs().iter().all(|i| self.is_input_in_pool(i)) {
            // tx is only dependent on outputs in pool
            if self.is_tx_valid(tx) {
                self.apply_tx(tx);
                handled.push(tx);
            }
        } else if tx
            .inputs()
            .iter()
            .any(|i| tx_set.contains(&i.output_tx_hash()))
        {
            // tx is dependent on some outputs from this batch
            dependent.push(tx)
        }
    }

    (handled, dependent)
}
\end{lstlisting}

\pagebreak
The \texttt{apply\_tx} function takes a valid transaction and mutates internal
state according to the transaction.

\begin{lstlisting}[language=Rust, style=boxed, caption={apply\_tx}]
/// Applies given tx to the internal pool
fn apply_tx(&mut self, tx: &Tx) {
    for input in tx.inputs().iter() {
        self.pool_mut().remove_utxo(&input_to_utxo(input));
    }
    for (i, output) in tx.outputs().iter().enumerate() {
        let utxo = UTXO::new(tx.hash(), i.try_into().unwrap());
        self.pool_mut().add_utxo(utxo, &output)
    }
}
\end{lstlisting}

\subsubsection*{Handler}\label{handler}

Handler implements the FIITcoin chain logic of validating, applying and 
returning valid subset of transactions. It's functionality is very simple,
loop until there are no dependent transactions, which are retrieved with
\texttt{handle\_independent} function. At the end returns the subset of
valid, applied transactions.


\begin{lstlisting}[language=Rust, style=boxed, caption={Handler::handle}]
fn handle(&mut self, possible_txs: Vec<&'a Tx>) -> Vec<&'a Tx> {
    let mut handled: Vec<&'a Tx> = vec![];
    let mut to_handle = possible_txs;

    loop {
        let (independent, dependent) = self.handle_independent(to_handle);
        handled.extend(independent);
        if dependent.is_empty() {
            break;
        }
        to_handle = dependent;
    }

    handled
}
\end{lstlisting}

\subsubsection*{MaxFeeHandler}

This implementation of a handler processes transactions in order to maximize
collected fees. Since there is no maximum count of processed transactions (if
there was the problem would be NP hard and very similiar to Knapsack Problem\cite{wiki:Knapsack_problem}),
I choose very simple heuristic how to achieve this\dots Process all valid
transactions.

Firstly, for each transaction a fee is calculated. The fee is a difference between
sum of input values and sum of output values.

\[ Fee = \sum_{n=1}^{|inputs|} inputs_n - \sum_{n=1}^{|outputs|} outputs_n \]

Then the initial list of proposed transactions is sorted by their fee, from
highest to lowest. And then are processed just like in \ref{handler}.

\begin{lstlisting}[language=Rust, style=boxed, caption={MaxFeeHandler::handle}]
fn handle(&mut self, possible_txs: Vec<&'a Tx>) -> Vec<&'a Tx> {
    let tx_map: HashMap<[u8; 32], &'a Tx> =
        possible_txs.iter().map(|&tx| (tx.hash(), tx)).collect();

    let mut with_fees: Vec<(u64, &Tx)> = possible_txs
        .iter()
        .filter_map(|&tx| match self.calc_fee(tx, &tx_map) {
            Some(fee) => Some((fee, tx)),
            None => None,
        })
        .collect();
    with_fees.sort_unstable_by(|tx1, tx2| tx1.0.cmp(&tx2.0));
    with_fees.reverse();

    let mut handled: Vec<&'a Tx> = vec![];
    let mut to_handle = with_fees.iter().map(|tx| tx.1).collect();

    loop {
        let (independent, dependent) = self.handle_independent(to_handle);
        handled.extend(independent);
        if dependent.is_empty() {
            break;
        }
        to_handle = dependent;
    }

    handled
}
\end{lstlisting}

\subsection{Tests}

Tests for this phase are in \emph{fiitcoin/tests}, and are implemented according
to provided test names in assignment.

Test 1 through 7 are in
\emph{fiitcoin/tests/is\_tx\_valid\_test.rs}. However test case number 7 isn't
implemented. This is because in the provided source codes, Java's \texttt{double}
is used, which is a signed representation of a fractional number\cite{oracleDOUBLEPRECISION}.
I used Rust's \texttt{u32}\cite{rustlangu32} to denominate value of an output,
thus testing if a sum of all outputs is negative is meaningless. Even if I created
a raw bytes representation of a transaction, and instead of unsigned integer
encoded a signed one, it would be treated as a large unsigned integer, in which
case the check for sum of inputs being less then or equal to sum of outputs would
fail, thus creating the same result.

Test 8 through 15 are in \emph{fiitcoin/tests/handler\_test.rs}. And max fee
tests 1 through 3 are in \emph{fiitcoin/tests/max\_fee\_handler\_test.rs}.

To run these tests, navigate to \emph{fiitcoin} and run this command:

\begin{verbatim}
$ RUST_LOG=debug cargo test --release
\end{verbatim}
