\section{Phase 2 - Trust and Consensus}

\subsection{Assignment}

Implement algorithm for distributed consensus based on relationalship graph.
Network is an oriented graph, in which each edge represents a trust relationalship
between two nodes. If there is an edge $A \rightarrow B$ it means that node $B$
is a follower of node $A$ ($A$ is a followee of $B$) and listens to transactions
proposed by $A$.

Nodes in network are either trusted or byzantine. Each trusted node should
reach consensus with other peers in the network. Implement trusted node,
which defines how each trusted node in network behaves. Then test this network
in a simulation with different relationalship graphs and with different parameters.
At the end of the simulation, each node should return the same subset of transactions,
upon which consensus was reached. Assume all transactions are valid. Different
simulation parameters:

\begin{itemize}
    \item Probability of an edge existing = $p\_graph \in \{0.1, 0.2, 0.3\}$,
    \item Probability of an node being byzantine = $p\_byzantine \in \{0.15, 0.3, 0.45\}$,
    \item Probability of a transaction distribution to node = $p\_tx\_dist \in \{0.01, 0.05, 0.1\}$,
    \item Number of rounds in simulation = $rounds \in \{10, 20\}$.
\end{itemize}

Also implement a byzantine node. This node acts like an adversary in the network,
and is trying to distrupt the network. There can up to 45\% of byzantine nodes
in the network. Their behaviour is:

\begin{itemize}
    \item Dead - don't resend any transactions,
    \item Selfish - only resend their transactions,
    \item Mix - switch between previous two behaviours.
\end{itemize}

At the end of the simulation, all trusted nodes must return the same subset of
transactions. The size of this subset should be maximal, and time to reach
consensus should be reasonable.

\subsection{Implementation}

Implementation for this phase is in submodule \emph{consensus}. This submodule is
also a Rust library crate, because I didn't know if I will reuse it later or not.

Implementations of trusted and byzantine node are in \emph{consensus/src/node.rs},
and both of them share common trait \texttt{Node}:

\begin{lstlisting}[language=Rust, style=boxed, caption={Node trait}]
pub trait Node<const N: usize> {
    /// If `ith` entry is `true` then this Node follows the `ith` Node
    fn followees_set(&mut self, followees: [bool; N]);

    /// Initializes proposed set of txs
    fn pending_txs_set(&mut self, pending_txs: HashSet<Tx>);

    /// Returns proposed txs, which shall be send to this Node's followers.
    /// After final round, the behaviour changes and it will return txs,
    /// on which consensus was reached.
    fn followers_send(&self) -> &HashSet<Tx>;

    /// Candites from different Nodes
    fn followees_receive(&mut self, candidates: &Vec<Candidate>);

    fn is_byzantine(&self) -> bool;
}
\end{lstlisting}

\subsubsection*{TrustedNode}

Trusted node starts with transmitting its transactions, and after receiving
a set of proposed transactions from followees it starts to track how many
distinct peers proposed each transaction, and adds each proposed transaction
to the ones it will transmit in next round.

A trusted node will assume a consensus was reached upon a transaction, if certain
number of peers propose it back to it. The number is called
\texttt{consensus\_threshold} in code and is calculated with this formula:

\[ CT = min(1, Probable followers - Probable byzantine nodes) \]

Each node has access to simulation's parameters so it can compute how many
followers it should have and how many byzantine nodes there are in network.

At the end of a simulation, each node returns the set of transactions upon
which consensus was reached on.


\subsubsection*{ByzantineNode}

Byzantine node is not interesting. It either doesn't resend any transactions,
only resends its transactions and no other, or switches between these two
behaviours.

\subsection{Tests}

There is only one test, and that is in \emph{consensus/tests/simulation.rs},
in which for all permutations of the simulation parameters are parallelly tested.
Each simulation has a result, and this result is written to \emph{/tmp/sim-result.txt}
file (in \emph{consensus}, there is a file from this test that I ran).

Since randomness is used to initialize each simulation, if it fails it is rerun,
but not more than 3 times, in order to make sure that it failed due to a bug, not
due to an edge case.

Each line in the \emph{consensus/sim-result.txt} contains what parameters were used,
how long the initialization of the simulation took, what seeds were used for randomness,
how long the simulation took and what was the size of the transaction subset upon
which consensus was reached on. If the simulation fails more than 3 times, there
would an additional entry in the line, which mentions this, but I didn't encounter
any.

Test cases in assignment are covered by this one simulation which uses all possible
permutations of parameters.

To run these tests, navigate to \emph{consensus} and run this command:

\begin{verbatim}
RUST_LOG=info cargo test --release
\end{verbatim}
