\section{Phase 3 - Blockchain}

\subsection{Assignment}

In this phase, create a node which is part of a distributed consensus protocol
based on UTXO blockchain. Implemented code will receive incoming transactions and
blocks, and maintain and update them. Reuse code from Phase 1(\ref{phase1}).

Blockchain is responsible for maintaining a chain of blocks. Because whole
chain would be big, only keep few newest blocks.

In this chain, forks can occur, so it is needed to maintain corresponding
UTXO pool.

New genesis block will not be mined, ignore any new block which is trying to be
a genesis block. When a fork occurs, at the max height of the chain is the newest
block. If a reorg happens after a fork, don't put transactions from the shorter
branch into the mempool. Also fees are neglected.

Bonus for implementing a multisig transactions.

\subsection{Implementation}

Implementation for this phase is in submodule. The main
code is in \emph{blockchain/src/blockchain.rs}. In there is a struct called
\texttt{Blockchain} which contains a ring buffer holding the most recent
blocks, and a mempool for unprocessed transactions. For manipulation with
the \texttt{Blockchain} struct, \texttt{BlockHandler} from \emph{blockchain/src/handler.rs}
should be used.

\pagebreak
\texttt{BlockHandler} has two methods for creating blocks:

\begin{lstlisting}[language=Rust, style=boxed, caption={BlockHandler}]
impl BlockHandler {
    pub fn create_block(&self, address: &VerifyingKey<Sha256>) -> Block {
        let parent = self.chain.block_at_max_height();
        let mut new_b = IncompleteBlock::new(parent.hash(), address);

        let utxo_pool = self.chain.utxo_pool_at_max_height();
        let mut handler = Handler::new(utxo_pool.clone());

        let tx_pool = self.chain.tx_pool_at_max_height();
        let txs = tx_pool.txs();
        let handled = handler.handle(txs);

        for &tx in handled.iter() {
            new_b.add_tx(tx.clone());
        }
        new_b.finalize()
    }

    pub fn create_fork(
        &self,
        parent_hash: [u8; 32],
        address: &VerifyingKey<Sha256>,
    ) -> Option<Block> {
        let (parent, utxo_pool) = self.chain.at_block_hash(parent_hash)?;
        let mut new_b = IncompleteBlock::new(parent.hash(), address);
        let mut handler = Handler::new(utxo_pool.clone());

        let tx_pool = self.chain.tx_pool_at_max_height();
        let txs = tx_pool.txs();
        let handled = handler.handle(txs);

        for &tx in handled.iter() {
            new_b.add_tx(tx.clone());
        }
        Some(new_b.finalize())
    }
}
\end{lstlisting}

\texttt{create\_block} is used when a new block is appended after linearly
to the chain, and \texttt{create\_fork} is used when a fork in chain is created.
Each of these two functions reuses the handler from Phase 1(\ref{phase1}),
and only bundles valid transactions into new block.

Then after creating a block, it can be added with \texttt{BlockHandler::process\_block},
which calls method \texttt{Blockchain::add\_block}:

\begin{lstlisting}[language=Rust, style=boxed, caption={Blockchain::add\_block}]
impl Blockchain {
    pub fn add_block(&mut self, block: Block) -> bool {
        let node = match self.at_block_hash(block.prev()) {
            Some(parent) => parent,
            None => return false,
        };

        let mut handler = fiitcoin::handler::Handler::new(node.1.clone());
        let txs: Vec<&fiitcoin::tx::Tx> = block.txs().iter().map(|tx| tx).collect();

        if handler.handle(txs).len() != block.txs().len() {
            log::warn!("Block contained invalid txs!");
            return false;
        };

        for tx in block.txs().iter() {
            self.mempool.remove(tx.hash());
        }
        self.chain.push((block, handler.move_pool()));

        true
    }
}
\end{lstlisting}

This method validates that the parent of the proposed block exists in the buffer,
if yes, once again validates all transactions, updates state of the mempool, and
appends the block to the chain.

\subsection{Tests}

Tests for this phase are in \emph{blockchain/tests/tests.rs}. Some tests cover
more than one test case from assignment, they have a comment above them
mentioning which tests they cover. To run them navigate to \emph{blockchain}
and run this command:

\begin{verbatim}
RUST_LOG=debug cargo test --release 
\end{verbatim}

\subsection{Multisig bonus}

The multisig implementation is in \emph{multisig}. I copied all needed files
from previous phases there, because I didn't want to break any existing tests.

I changed transaction outputs from having only one verifier, to holding a list
of verifiers and having a threshold indicating how many valid signatures are
needed to unlock given output. Also each transaction input has a list of signatures
which are used to unlock corresponding output.

Tests are in \emph{multisig/tests}, in \emph{multisig/tests/multisig\_tx.rs} are
tests just testing individual transactions and in
\emph{multisig/tests/multisig\_blockchain.rs} are tests for using these multisig
transactions in blockchain.
